\section{Comparison to other models}
\begin{frame}
\frametitle{Versus}
\begin{itemize}
\item Petri nets: places (conditions), transitions (events), arcs between places$\leftrightarrow$transitions \textrightarrow net of pre-/postconditions of possible events \\
\textrightarrow any number of tokens in the net, tokens trigger transitions and are transferred to postconditions \\
\textrightarrow actors modeled as units of computation, petri nets as transitions/events mostly independent of units
\end{itemize}
\end{frame}

\begin{frame}
\frametitle{Versus}
\begin{itemize}
\item process calculi: collection of formal models, processes interact by message-passing (anonymous, through channels), processes as composition of primitives and operators subject to algebraic laws \\
\textrightarrow message-passing similar to actors, processes could be modeled to resemble actors
\item I/O automata: automaton modeled into any kind of single component, as state machine with in-/output and (hidden) internal actions \\
\textrightarrow similar to encapsulation in actors, automata can be modeled to resemble actor model structure
\end{itemize}
\end{frame}

\begin{frame}
\frametitle{Conclusion?}
\begin{itemize}
\item actor model has great potential in achieving scalable concurrency
\item quantitative and qualitative developer support is lacking, but increasing
\item more and more use in large and small, distributed and embedded, commercial and free/open projects
\end{itemize}
\end{frame}